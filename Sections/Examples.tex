\section{Examples}

\begin{frame}[headless]{}
	\sectionpage
\end{frame}

\begin{frame}{Simple Slide}
	This is a simple slide.
\end{frame}

\begin{frame}[noframenumbering]{No Slide Numbering}
	This slide is not numbered and is citing reference \cite{knuth74}.
\end{frame}

\begin{frame}{Blocks}
	These blocks are part of 1 slide, to be displayed consecutively.
	\begin{block}{Block}
		Text.
	\end{block}
	\pause % Automatically creates a new "page" split between the above and above + below
	\begin{alertblock}{Alert block}
		Alert \alert{text}.
	\end{alertblock}
	\pause % Automatically creates a new "page" split between the above and above + below
	\begin{exampleblock}{Example block}
		Example Text
	\end{exampleblock}
\end{frame}

\begin{frame}{Columns}
	\begin{columns}
		\column{0.45\textwidth}
			This text appears in the left column and wraps neatly with a margin between columns.
		
		\column{0.45\textwidth}
			\includegraphics[width=\linewidth]{Assets/placeholder.jpg}
	\end{columns}
\end{frame}

\begin{frame}{Lists}
	\begin{columns}[T, onlytextwidth] % T for top align, onlytextwidth to suppress the margin between columns
		\column{0.33\textwidth}
			Items:
			\begin{itemize}
				\item Item 1
				\begin{itemize}
					\item Subitem 1.1
					\item Subitem 1.2
				\end{itemize}
				\item Item 2
				\item Item 3
			\end{itemize}
		
		\column{0.33\textwidth}
			Enumerations:
			\begin{enumerate}
				\item First
				\item Second
				\begin{enumerate}
					\item Sub-first
					\item Sub-second
				\end{enumerate}
				\item Third
			\end{enumerate}
		
		\column{0.33\textwidth}
			Descriptions:
			\begin{description}
				\item[First] Yes.
				\item[Second] No.
			\end{description}
	\end{columns}
\end{frame}



\begin{frame}{Blocks}
    \begin{itemize}
        \item This template uses tcolorboxes for callouts, and has a custom enviroment defined for colored boxes.
    \end{itemize}
    \begin{columns}[T, onlytextwidth]
        \setlength{\columnsep}{2pt} % Reduce the space between columns
        \begin{column}{0.48\textwidth}
            \begin{block}{Regular Block}
                You can use regular blocks for callouts like this.
            \end{block}
        \end{column}
        \begin{column}{0.48\textwidth}
            \begin{colorblock}[slideBG]{blockColor}{Color Block}
                You can use  color blocks for callouts like this.
            \end{colorblock}
        \end{column}
    \end{columns}
\end{frame}

\begin{frame}[fragile]{Tables}
    \begin{itemize}
        \item To put a grey box behind a table, use the following:
    \end{itemize}
    \begin{mintedblock}[latex]{Grey Tables}
        \begin{table}
            \begin{table}
            \begin{greytablebox}
                \begin{tabular}{}
                    % Your table stuff here
                \end{tabular}
            \end{greytablebox}
        \end{table}
    \end{mintedblock}
\end{frame}


\begin{frame}[headless,c]{Headless}
There is a custom frame option, \texttt{headless}, that removes the frame title.
\end{frame}


\begin{frame}[fragile]{Code}
    There are two ways to display code - this is the block...
    \begin{mintedblock}[c]{Mintedblock}
#include <stdio.h>
int main() {
	// printf() displays the string inside quotation
	printf("Hello, World!");
	return 0;
}
    \end{mintedblock}
 ... and this is the listing.
    \begin{mintedlisting}[rust]
fn main() {
	println!("Hello World!");
}
    \end{mintedlisting}
\end{frame}

\begin{frame}{Font Tests}
	This is regular text.\\
	\textbf{This is bold text.}\\
	\textit{This is italic text.}\\
	\texttt{This is monospaced text.}
  \end{frame}